\documentclass{article}
\usepackage[style=ieee]{biblatex}
\bibliography{references.bib}
\usepackage[utf8]{inputenc}
\usepackage{graphicx}
\usepackage{float}
\usepackage{listings}
\usepackage{algorithm}
\usepackage{algpseudocode}
\usepackage{amsmath}
	
\title{Evolutionary Algorithm for the Bonded Molecular Conformation Problem}
\author{Bennett Gale}
\date{August 2018}
\lstset{language=C++, basicstyle=\ttfamily}

\begin{document}
\maketitle
\tableofcontents
\pagebreak

\section{Introduction}

The model molecular energy minimisation problem involves the geometric
configuration of a chain of atoms in two dimensions, each connected by rigid
bonds of unit length, such that the total potential energy yielded by the sum of
individual Lennard-Jones potentials is globally minimised. The total potential
energy of the cluster is given by 
$$V(r)=\sum^{N-1}_{i=1}\sum^{N}_{j=i+1}\left(r^{-12}_{ij}-2r^{-6}_{ij}\right)$$
Where $r$ is the euclidean distance between a pair of atoms and $N$ is the
number of atoms in the chain.

The energy minimisation problem is historically difficult because of the
exponential relationship between the locally stable structures and the number of
atoms, $N$, in the molecule. That is, as the size of the cluster increases, the
number of local minima increases exponentially such that when $N$ approaches
100, there exist more than $10^{140}$ minima \cite{DAVEN1996195}. It is
therefore necessary to employ sophisticated search algorithms in order to make
larger input sizes attainable.

Section~\ref{litreview} will discuss some of the research that has already taken
place in this area. Section~\ref{proposed} will detail the algorithm proposed by
this paper.

\section{Literature Review} \label{litreview}

Genetic algorithms have been applied \cite{doi:10.1002/qua.560440214,
PULLAN1998331, } successfully in order to extend the range of input size $N$ that
is reachable through structural optimisation. This population based approach
emulates the Darwinian process of natural selection, and works by performing
crossover operations on pairs of candidate configurations with selection
probabilities based on their fitness, producing new child nodes which contain
genetic material from both parents. With every crossover operation there is a
small chance of subsequent random mutation of the child configurations, the
purpose of which is to promote genetic diversity within the population and
prevent premature convergence. Over many generations the population converges
towards a global minimum as each configuration communicates information about
the global search space through the crossover operation. This method has been
successful in producing results that were previously unseen in literature for
certain input sizes \cite{PULLAN1998331}.

Judson \textit{et al.} compare the performance of GAs with simulated annealing,
a probabilistic global optimisation technique in which Monte Carlo steps are
taken and new states are accepted if they are fitter, or with a probability
based on a function of a temperature value \cite{doi:10.1002/qua.560440214}.
When applied to the molecular conformation problem, they found that the GA
method and the SA method, both applied as global meta-heuristics in conjunction
with gradient-based local optimisers (conjugate gradient), similarly
outperformed purely stochastic methods \cite{doi:10.1002/qua.560440214}. Judson
\textit{et al.} classify points on the model energy surface as being in one of
three states: knotted and of high energy, unknotted and of low but not minimal
energy, and unknotted with globally minimised energy
\cite{doi:10.1002/qua.560440214}. They applied both GA and SA in order to make
the transition from the first state to the second state before applying local
optimisation to descend to the bottom of the basin.

Genetic operators for the two dimensional bonded Lennard-Jones problem have
since been further developed with successful results \cite{PULLAN1998331},
proving the GA approach to be a viable global optimisation technique. Pullan
\cite{PULLAN1998331} proposes three genetic crossover methods and four mutation
operators and applies these in a parallel genetic algorithm to find global
minima for most values of $N, 2 \leq N \leq 55$, as well as a minimum for $N =
61$ which was previously unseen in literature.

This two stage hybrid method has been successful, but Judson \textit{et al.}
also suggest the possibility of adding a third optimisation stage which is
intermediate between the global and local gradient methods
\cite{doi:10.1002/qua.560440214}. It is thought that a non-gradient optimisation
method such as the simplex method may provide a means of stepping over barriers
into possibly deeper minima surrounding those found by the global method in the
first phase.

More recently, the Conformational Space Annealing Method
\cite{PhysRevLett.91.080201} has been developed as a global optimisation
algorithm that has produced very good results when applied to the problem of an
unbound cluster of atoms with Lennard-Jones relationships. This algorithm
combines concepts of Monte Carlo with minimisation, genetic algorithms, and
simulated annealing. While this technique has not been applied to the bonded
variant of the Lennard-Jones conformation problem as discussed here, it
demonstrates that a hybrid approach combining SA and evolutionary population
based algorithms may be a viable technique in a similar problem domain.

\section{Proposed Algorithm} \label{proposed}

As discussed in section~\ref{litreview}, global meta-heuristics such as the
genetic algorithm, have been used successfully as a means of navigating the
global search space and avoiding local minima that are caused by knots in the
configurations \cite{PULLAN1998331, doi:10.1002/qua.560440214}. These methods
use the global optimiser to find unknotted potential candidate configurations
and then refine them using gradient-based methods. Judson \textit{et al.} have
suggested a three stage system in which a non-gradient optimiser such as the
simplex method may be used as an intermediate stage between the global and
local-gradient optimisation in order to traverse barriers into potentially
deeper basins \cite{doi:10.1002/qua.560440214}.

As simulated annealing has been found to be successful when combined with
evolutionary algorithms in similar problem domains \cite{PhysRevLett.91.080201},
this paper will explore its use as an intermediate non-gradient method such that
the proposed algorithm has three levels: global meta-heuristic (GA),
intermediate non-gradient optimisation (SA) and local gradient-based
optimisation (BFGS). The performance of this three stage optimisation method
will be compared with a purely evolutionary algorithm without the intermediate
step.

\subsection{Genetic Encoding}

Encoding each  molecule as a chain of two-dimensional cartesian coordinates
would require constraints to be enforced so that the relative orientations and
unit length bonds are maintained. In order to avoid this, therefore, molecules
will be encoded as sequences of $N-2$ angles, where $N$ is the number of atoms
in the molecule and each angle, $\alpha_i$ represents the rotation of the
segment following atom $a_i$. This encoding scheme means that cartesian
coordinates must be calculated at each Lennard-Jones potential computation as it
requires euclidean distances between pairs of atoms to be computed, but results
in much simpler crossover and mutation operations.

\subsection{Selection \& Crossover Method}

An elitist methodology will be utilised throughout the GA optimisation, meaning
that the single fittest configuration will be passed on unmodified to subsequent
generations in order to preserve beneficial traits and provide a mechanism for
remembering information about the best configuration that has been found so far.

Pairs of parents to be crossed over will then be selected using the Roulette
method, where the probability for a potential parent to be selected is based on
a function of its fitness value. The `elite' configurations that were inherited
from the previous generation will also be candidates for selection using this
method. Selected parents will undergo one of the following crossover operators
with equal probabilities of occurring:

\subsubsection{Single-Point Crossover}

A single atom $a_i$ will be randomly selected. Atoms to the left of this point
in one parent will form the left segment of the child, and the right segment of
the child will be similarly formed from the atoms to the right of the second
parent. A second child may be generated by repeating with the opposite segments
(the right segment from parent one and the left segment from parent two).
Finally, the angle $\alpha_i$ at the crossover point will be iterated through
discrete steps until an angle is found which results in the lowest energy, as
described by Pullan \cite{PULLAN1998331}.

\subsubsection{Two-Point Crossover}

Two crossover points will be selected at random and the molecules will be
combined similarly to the single-point crossover, with each child containing two
segments from one parent and one segment from the other. As with the single
point crossover, the angles of both crossover points will be optimised.

\subsection{Mutation and Intermediate Optimisation}

The mutation mechanism is where the intermediate non-gradient optimisation
algorithm will be incorporated. In traditional genetic algorithms, mutation has
a small chance of occurring and is random, resulting in some states which are
less fit in the hopes that this will later lead to a deeper basin in the search
space. By incorporating simulated annealing at the mutation stage it is hoped
that the configuration will be dragged in a downward direction while also
providing enough stochastic properties to allow boundaries between minima to be
traversed, which is the goal of any mutation operator.

Mutation will have a low chance of occurring so as not to corrupt fit
configurations too frequently and lose information provided by the global
search. Each generation, child nodes that are produced during the crossover
stage will have a small chance of becoming initial states in the simulated
annealing search.

\subsubsection{Annealing Schedule}

During annealing random angles are selected and modified by random amounts. If
the change produces an improved state, it is accepted. If the change produces a
worsened state, it is accepted with a probability of $1 - e^{\delta E / kT}$
such that the current `temperature' as well as how bad the state is determines
whether it is accepted.

The goal at this intermediate stage is not to find the global minimum, but
merely to provide a more localised search for deeper basins in the vicinity of
the current configuration; it is hoped that if the GA search has found a local
minimum in close proximity to the global minimum, the SA stage will take large
enough steps in order to climb over the boundary into the neighbouring basin
where it can be finally optimised by the gradient search in the next stage.

Therefore, an annealing schedule that prioritises speed over optimality will be
chosen.

\subsection{Local Optimisation}

The final stage of the algorithm, after a candidate configuration has been found
by the GA and its immediate vicinity has been further explored through mutation,
is to refine the configuration using gradient-based local search. This stage
will use the gradient of the objective function in order to guarantee that the
bottom of the current basin will be reached. The quasi-Newton method
Broyden–Fletcher–Goldfarb–Shanno (BFGS) will be applied at this stage resulting
in a fast descent towards the current local minimum at each stage.

The coordinates of this locally optimised configuration will also be passed back
to the GA. This is known as a `Lamarkian' process and has been found to be
beneficial when used with this problem \cite{doi:10.1002/qua.560440214}.

%\subsection{Implementation}

% \begin{algorithm} \caption{Proposed High-Level Algorithm} \label{algorithm}
%   \centering \begin{algorithmic} \Procedure{optimise}{} % \State $nn \gets new
%   Network$ %   \For{$i \gets 0; i <= number\_of\_epochs; i++$} %
%   \For{$\textit{each data partition of minibatch size}$} %           \State
%   $\textit{Do forward pass}$ %           \State $\textit{Backpropagate the
%   resulting error}$ %           \State $\textit{Update the weights
%   proportionally to their contribution to the error}$ %       \EndFor %
%   \State $\textit{Shuffle and repartition the data}$ %   \EndFor \State $P
%   \gets \textit{Initial Population Size}$ \State $population \gets initialise$
%   \EndProcedure \end{algorithmic}
	
% \end{algorithm}

\pagebreak

\printbibliography

\end{document}
\documentclass{article}
\usepackage[style=ieee]{biblatex}
\bibliography{references.bib}
\usepackage[utf8]{inputenc}
\usepackage{graphicx}
\usepackage{float}
\usepackage{listings}
\usepackage{algorithm}
\usepackage{algpseudocode}
\usepackage{amsmath}
	
\title{Lennard-Jones Energy Minimisation}
\author{Bennett Gale}
\date{July 2018}
\lstset{language=C++, basicstyle=\ttfamily}

\begin{document}
\maketitle
\tableofcontents
\pagebreak

\section{Introduction}

The model molecular energy minimisation problem involves the geometric
configuration of a cluster of atoms in two dimensions, each connected by rigid
bonds of unit length, such that the total potential energy yielded by the sum of
individual Lennard-Jones potentials is globally minimised. The total potential
energy of the cluster is given by 
$$V(r)=\sum^{N-1}_{i=1}\sum^{N}_{j=i+1}\left(r^{-12}_{ij}-2r^{-6}_{ij}\right)$$
Where $r$ is the euclidean distance between a pair of atoms and $N$ is the
number of atoms in the chain.

The energy minimisation problem is historically difficult because of the
exponential relationship between the locally stable structures and the number of
atoms, $N$, in the molecule. That is, as the size of the cluster increases, the
number of local minima increases exponentially such that when $N$ approaches
100, there exist more than $10^{140}$ minima \cite{DAVEN1996195}. It is
therefore necessary to employ sophisticated search algorithms in order to make
larger input sizes attainable.

Section~\ref{litreview} will discuss some of the research that has already taken
place in this area. Section~\ref{proposed} will detail the algorithm proposed by
this paper.

\section{Literature Review} \label{litreview}

Genetic algorithms have been applied \cite{doi:10.1002/qua.560440214,
PULLAN1998331} successfully in order to extend the range of input size $N$ that
is reachable through structural optimisation. This population based approach
emulates the Darwinian process of natural selection, and works by performing
crossover operations on pairs of candidate configurations with selection
probabilities based on their fitness, producing new child nodes which contain
genetic material from both parents. With every crossover operation there is a
small chance of subsequent random mutation of the child configurations, the
purpose of which is to promote genetic diversity within the population and
prevent premature convergence. Over many generations the population converges
towards a global minimum as each configuration communicates
information about the global search space through the crossover operation. This
method has been successful in producing results that were previously unseen in
literature for certain input sizes.

Judson \textit{et al.} compare the performance of GAs with simulated annealing,
a stochastic global optimisation technique in which Monte Carlo steps are taken
and new states are accepted if they are fitter, or with a probability based on a
function of a temperature value \cite{doi:10.1002/qua.560440214}. When applied
to the molecular conformation problem, they found that the GA method and the SA
method, both applied as global meta-heuristics in conjunction with
gradient-based local optimisers (conjugate gradient), similarly outperformed
purely stochastic methods \cite{doi:10.1002/qua.560440214}. Judson \textit{et
al.} classify points on the model energy surface as being in one of three
states: knotted and of high energy, unknotted and of low but not minimal energy,
and unknotted with globally minimised energy \cite{doi:10.1002/qua.560440214}.
They applied both GA and SA in order to make the transition from the first state
to the second state before applying local optimisation to descend to the bottom
of the basin.

Genetic operators for the two dimensional bonded Lennard-Jones problem have
since been further developed with successful results \cite{PULLAN1998331},
proving the GA approach to be a viable global optimisation technique. Pullan
\cite{PULLAN1998331} proposes three genetic crossover methods and four mutation
operators and applies these in a parallel genetic algorithm to find global
minima for most values of $N, 2 \leq N \leq 55$, as well as a minimum for $N =
61$ which was previously unseen in literature.

This two stage hybrid method has been successful, but Judson \textit{et al.}
also suggest the possibility of adding a third optimisation stage which is
intermediate between the global and local gradient methods
\cite{doi:10.1002/qua.560440214}. It is thought that a non-gradient optimisation
method such as the simplex method may provide a means of stepping over barriers
into possibly deeper minima surrounding those found by the global method in the
first phase.

More recently, the Conformational Space Annealing Method
\cite{PhysRevLett.91.080201} has been developed as a global optimisation
algorithm that has produced very good results when applied to the problem of an
unbound cluster of atoms with Lennard-Jones relationships. This algorithm
combines concepts of Monte Carlo with minimisation, genetic algorithms, and
simulated annealing. While this technique has not been applied to the bonded
variant of the Lennard-Jones conformation problem as discussed here, it
demonstrates that a hybrid approach combining SA and evolutionary population
based algorithms may be a viable technique in a similar problem domain.

\section{Proposed Algorithm} \label{proposed}

As discussed in section~\ref{litreview}, global meta-heuristics such as the
genetic algorithm, have been used successfully as a means of navigating the
global search space and avoiding local minima that are caused by knots in the
configurations \cite{PULLAN1998331, doi:10.1002/qua.560440214}. These methods
use the global optimiser to find unknotted potential candidate configurations
and then refine them using gradient-based methods. Judson \textit{et al.} have
suggested a three stage system in which a non-gradient optimiser such as the
simplex method may be used as an intermediate stage between the global and
local-gradient optimisation in order to traverse barriers into potentially
deeper basins \cite{doi:10.1002/qua.560440214}.

As simulated annealing has been found to be successful when combined with
evolutionary algorithms in similar problem domains \cite{PhysRevLett.91.080201},
this paper will explore its use as an intermediate non-gradient method such that
the proposed algorithm has three levels: (1) global meta-heuristic, (2),
intermediate non-gradient optimisation and (3) local gradient-based
optimisation.



\pagebreak

\printbibliography

\end{document}
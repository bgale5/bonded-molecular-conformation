\documentclass{article}
\usepackage[style=ieee]{biblatex}
\bibliography{references.bib}
\usepackage[utf8]{inputenc}
\usepackage{graphicx}
\usepackage{float}
\usepackage{listings}
\usepackage{algorithm}
\usepackage{algpseudocode}
\usepackage{amsmath}
	
\title{Lennard-Jones Energy Minimisation}
\author{Bennett Gale}
\date{July 2018}
\lstset{language=C++, basicstyle=\ttfamily}

\begin{document}
\maketitle
\tableofcontents
\pagebreak

\section{Introduction}

The model molecular energy minimisation problem involves the geometric
configuration of a cluster of atoms in two dimensions, each connected by rigid
bonds of unit length, such that the total potential energy yielded by the sum of
individual Lennard-Jones potentials is globally minimised. The total potential
energy of the cluster is given by 
$$V(r)=\sum^{N-1}_{i=1}\sum^{N}_{j=i+1}\left(r^{-12}_{ij}-2r^{-6}_{ij}\right)$$
Where $r$ is the euclidean distance between a pair of atoms and $N$ is the
number of atoms in the chain.

The energy minimisation problem is historically difficult because of the
exponential relationship between the locally stable structures and the number of
atoms, $N$, in the molecule. That is, as the size of the cluster increases, the
number of local minima increases exponentially such that when $N$ approaches
100, there exist more than $10^{140}$ minima \cite{DAVEN1996195}. It is
therefore necessary to employ sophisticated search algorithms in order to make
larger input sizes attainable.

Section~\ref{litreview} will discuss some of the research that has already taken
place in this area. Section~\ref{proposed} will detail the algorithm proposed by
this paper.

\section{Literature Review} \label{litreview}

Biased modelling approaches, in which prior assumptions are made regarding the
structure of optimal configurations, have been used successfully in the past to
yield powerful results. Initially, for example, global minima for many of
$N\leq150$ were found using lattices derived from icosahedron
\cite{doi:10.1063/1.453492}. However while these methods can be the most
powerful, they suffer in cases when the assumed model is largely different from
the real configuration \cite{doi:10.1002/jcc.20096}. Therefore, more general,
less biased algorithms have been applied in order to increase robustness against
such exceptional cases.

Genetic algorithms have been employed in in order to successfully extend the
range of input size $N$ that is reachable through structural optimisation
\cite{DAVEN1996195, doi:10.1002/qua.560440214,PULLAN1998331}. This population
based approach emulates the Darwinian process of natural selection, and works by
performing crossover operations on pairs of candidate configurations over many
generations until a global minimum is produced. This method has been successful
in producing results that were previously unseen in literature for certain $N$
sizes, due to its generality when compared with more biased approaches.

The Conformational Space Annealing Method is a more recent unbiased global
optimisation algorithm that has produced very good results
\cite{PhysRevLett.91.080201}. This technique is highly robust against random
initial configurations thanks to its generality. As a result, it has therefore
been able to produce all global minima for $n\leq183$ without suffering from
exceptional cases like those experienced by the biased modelling algorithms.
This algorithm combines concepts of Monte Carlo with Minimisation, Genetic
Algorithm and simulated Annealing, and requires no specific knowledge or
assumptions regarding the optimal structure. Local minimisation is performed on
a set of configurations, and then similarly to the mating process of genetic
algorithm, a subset of configurations is produced. A parameter $D_cut$ is
introduced and acts as a temperature variable in the annealing process. This
parameter represents a distance measure between configurations and is used to
maintain diversity in the population \cite{PhysRevLett.91.080201}.

Another unbiased global optimisation algorithm, known as dynamic lattice
searching (DLS), works by repeatedly constructing and searching a dynamic
lattice until a global minimum is produced \cite{doi:10.1002/jcc.20096}. This
lattice is built by searching the possible location sites for added atoms, which
are then iteratively relocated from lattice sites of higher energy to those of
lower energy. This method is very efficient because of how much it reduces the
search space, but remains unbiased and therefore does not exhibit the same
limitations seen in the biased modelling based techniques.

\section{Proposed Optimisation Algorithm} \label{proposed}

\pagebreak

\printbibliography

\end{document}